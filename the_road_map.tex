\section{The road map} 

Where do I plan to guide the reader? The shangri-la of this map is 
Papert's Principle:

\begin{quote}
    Some of the most crucial steps in mental growth are based
    not simply on acquiring new skills, but on acquiring new administrative ways to
    use what one already knows. \cite{artificial_intelligence:minsky__society_of_mind}
\end{quote}

This principle is the objective because I consider the fundamental aspect of
taste as the ability to organize current knowledge in efficient and beautiful
ways.

More specifically, I'll point to sources that present techniques to reach the
same goals of \cite{education:keller__the_risks_and_benefits_of_teaching_purely_functional}

\begin{quote}
    \begin{enumerate}
    \item Convey the elementary techniques of programming (the practical aspect). 
    \item Introduce the essential concepts of computing (the theoretical aspect). 
    \item Foster the development of analytic thinking and problem solving skills (the methodological aspect).
    \end{enumerate}
\end{quote}

\subsection{Styles and Stances}

What language to begin the journey? This is at the same time an important and a
meaningless question. 

It is important because all languages are designed with certain goals. My
recommendation could not be stated better than
\cite{education:norvig__teach_yourself_programming} does:

\begin{quote}
    \begin{itemize}
        \item Use your friends. When asked "what operating system should I use,
        Windows, Unix, or Mac?", my answer is usually: "use whatever your friends
        use." The advantage you get from learning from your friends will offset any
        intrinsic difference between OS, or between programming languages. Also
        consider your future friends: the community of programmers that you will be
        a part of if you continue. Does your chosen language have a large growing
        community or a small dying one? Are there books, web sites, and online
        forums to get answers from? Do you like the people in those forums?  
        
        \item Keep it simple. Programming languages such as C++ and Java are designed for
        professional development by large teams of experienced programmers who are
        concerned about the run-time efficiency of their code. As a result, these
        languages have complicated parts designed for these circumstances. You're
        concerned with learning to program. You don't need that complication. You
        want a language that was designed to be easy to learn and remember by a
        single new programmer.  
        \item Play. Which way would you rather learn to play
        the piano: the normal, interactive way, in which you hear each note as soon
        as you hit a key, or "batch" mode, in which you only hear the notes after
        you finish a whole song? Clearly, interactive mode makes learning easier for
        the piano, and also for programming. Insist on a language with an
        interactive mode and use it.  
    \end{itemize} 
    
    Given these criteria, my recommendations for a first programming language
    would be Python or Scheme.  But your circumstances may vary, and there are
    other good choices. If your age is a single-digit, you might prefer Alice or
    Squeak (older learners might also enjoy these). The important thing is that
    you choose and get started.  
\end{quote}

But why it is a meaningless choice? Because the true skill of a programmer is
not to understand one language or another. As \cite{education:papert_mindstorms}
puts it (he uses children, but they are just special example case):

\begin{quotation}
        By deliberately learning to imitate mechanical thinking, the learner becomes
    able to articulate what mechanical thinking is and what it is not. The
    exercise can lead to greater confidence about the ability to choose a
    cognitive style that suits the problem.  Analysis of "mechanical thinking"
    and how it is different from other kinds and practice with problem analysis
    can result in a new degree of intellectual sophistication. By providing a
    very concrete, down-to-earth model of a particular style of thinking, work
    with the computer can make it easier to understand that there is such a
    thing as a "style of thinking." And giving children the opportunity to
    choose one style or another provides an opportunity to develop the skill
    necessary to choose between styles. Thus instead of inducing mechanical
    thinking, contact with computers could turn out to be the best conceivable
    antidote to it. And for me what is most important in this is that through
    these experiences these children would be serving their apprenticeships as
    epistemologists, that is to say learning to think articulately about
    thinking.
\end{quotation}

Let me be more clear. The most important skill a great programmer possess is the
capacity to think at different levels, and to think about thinking, because this
develops the crucial mental flexibility of a good designer.
\cite{education:spolsky__the_perils_of_java_schools}
