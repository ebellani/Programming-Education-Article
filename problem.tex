\section{The Problem}

The greatest problem that I try to partially address in this work
is mentioned by \cite{futurism:kurzweil_singularity_is_near}:

\begin{quotation}
    Most education in the world today, including in the wealthier communities,
    is not much changed from the model offered by the monastic schools of
    fourteenth-century Europe. Schools remain highly centralized institutions
    built upon the scarce resources of buildings and teachers. The quality of
    education also varies enormously, depending on the wealth of the local
    community (the American tradition of funding education from property taxes
    clearly exacerbates this inequality), thus contributing to the have/have not
    divide.
\end{quotation}

Not only that, but the quality of the teaching experience, by yielding to short
term pressures of some market segments, are \emph{dumbing down} their courses so
they can ``adjust to the times''. \cite{education:spolsky__the_perils_of_java_schools}

With the advent of the Internet,  the potential for emergent learning arises,
both individually and communally, but in the sea of information exists a lot of
pitfalls, possibilities for lost time and motivation and confusion.


