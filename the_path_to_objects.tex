\subsection{Further readings or The path to objects}

\subsubsection{Concrete Abstractions}

This  work is aimed at the same audience as HTDP and SICP, but with a different
emphasis. It still uses the scheme language, which allows \htmladdnormallink{the
use of DrScheme}{http://gustavus.edu/+max/concabs/schemes/drscheme/}, a fact
that gives it the same benefits as HTDP in this aspect.

It is also a textbook that gets the epistemological importance of the teaching,
\htmladdnormallink{as this small description by its authors shows}{http://gustavus.edu/+max/concrete-abstractions.html}:

\begin{quote}
    \begin{enumerate}
        \item  The book features thorough integration of theory and practice,
        and presents theory as an essential component of practice, rather than
        in contrast to it. Thus, students are introduced to the analytic tools
        they need to write effective and efficient programs, in the context of
        practical and concrete applications.

        \item Significant programming projects are included that actively
        involve students in applying concepts. Each chapter ends with an
        application section, in which the concepts from that chapter are applied
        to a significant, interesting problem that involves both program design
        and modifying existing code.
        
        \item The authors present development of object-oriented programming,
        one concept at a time. Each of the component concepts that constitute
        object-oriented programming (OOP) is introduced independently; they are
        then incrementally blended together.

        \item In keeping with modern curricular recommendations, this book
        presents multiple programming paradigms: functional programming,
        assembly-language programming, and object-oriented programming--enabling
        the student to transition easily from Scheme to other programming
        languages.

        \item The final chapter provides a transition from Scheme to Java.
        Providing this transition within a single book allows Java to be
        explained by comparison with Scheme, including showing an example
        program in both languages. Java also supports exploration of
        event-driven graphical user interfaces and concurrency. 
    \end{enumerate}
\end{quote}

or consider this self description directly from \cite{education:hailperin_conabs}

\begin{quote}
       In summary, this book is designed to introduce you to how computer scientists
    think and work. We assume that as a reader, you become actively involved in reading
    and that you like to play with things. We have provided a variety of activities that
    involve hands-on manipulation of concrete objects such as paper chains, numbered
    cards, and chocolate candy bars. The many programming exercises encourage you to
    experiment with the procedures and data structures we describe. And we have posed
    a number of problems that allow you to play with the abstract ideas we introduce.

       Our major emphasis is on how computer scientists think, as opposed to what they
    think about. Our applications and examples are chosen to illustrate various problem-
    solving strategies, to introduce some of the major themes in the discipline, and to
    give you a good feel for the subject. 
\end{quote}

The use of a different mix of exercises, a different philosophy and the explicit
tackling of object orientation and a transition path from scheme to another
language makes this an choice to check instead or along SICP. Again, the
application of the design recipes is something that enriches the experience in
my opinion.  


\subsubsection{Smalltalk, Objects and Design}

I present this book as the best source to learn object orientation that I have
ever found. Considering that objects play such a huge importance on the software
industry in our times, I think a suggestion focused on that would be reasonable.

I think this book is very good because it also gets the epistemological stance
of \cite{education:papert_mindstorms}, as the following extracts from
\cite{programming:chamond__sod} show:

\begin{quote}
        The goal is to design more like veteran software do. They choose among
    alternatives quickly and subconsciously, drawing upon years of experience,
    something like chess grandmasters choosing among moves. Lacking this experience,
    novices have a hard time discovering plausible alternatives, and an impossible
    time discovering subtle ones. For their sake then, I often argue alternatives
    and the trade-offs between them, so that they will have an outside chance of
    considering the same design the veterans do.

        When you finish the book, I hope you will be able to think about software
    problems in some of the ways that the veterans do, and be able to implement your
    thoughts in Smalltalk. Not expertly, however. Mastery of both object-oriented
    design and Smalltalk comes only with actual practice. Of course, these are trims
    for nay activities, from driving a car to playing the piano. But Smalltalk,
    more than most software tools, requires you to plunge in and abandon yourself
    to the language and environment. A taste for adventure definitely helps, more
    that in learning how to drive a car.
\end{quote}
