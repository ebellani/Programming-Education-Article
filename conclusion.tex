\section{Conclusion}

I've tried to convey an adequate way that if followed would benefit a person and
prepare it to became a competent programmer, not only in the sense of being able
to dish out code, but to think at several abstraction levels simultaneously. 

What I wish to stress is that the \emph{mastery} of this concepts takes
time and practice \cite{programming:chamond__sod}, as much as a decade according to
\cite{education:norvig__teach_yourself_programming}. But to maintain a steady
progression one needs to develop a \emph{design taste}. That is what I hope the
path outlined here would provide.

To wrap the article, I wish to cite \cite{education:paul_graham__a_taste_for_makers} words on taste:

\begin{quotation}

    Mathematicians call good work ``beautiful'', and so, either now or in the
    past, have scientists, engineers, musicians, architects, designers, writers,
    and painters. Is it just a coincidence that they used the same word, or is
    there some overlap in what they meant? If there is an overlap, can we use
    one field's discoveries about beauty to help us in another?

    For those of us who design things, these are not just theoretical questions.
    If there is such a thing as beauty, we need to be able to recognize it. We
    need good taste to make good things. Instead of treating beauty as an airy
    abstraction, to be either blathered about or avoided depending on how one
    feels about airy abstractions, let's try considering it as a practical
    question: \emph{how do you make good stuff?}

\end{quotation}
