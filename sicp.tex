\subsection{The structure and interpretation of computer programs}

This book is chosen because the authors clearly understand Papert's principle
(they were both his students) by several comments as:


\begin{quotation}
    Our design of this introductory computer-science subject reflects two major
    concerns. First, we want to establish the idea that a computer language is not
    just a way of getting a computer to perform operations but rather that it is a
    novel formal medium for expressing ideas about methodology. Thus, programs must
    be written for people to read, and only incidentally for machines to execute.
    Second, we believe that the essential material to be addressed by a subject at
    this level is not the syntax of particular programming- language constructs, nor
    clever algorithms for computing particular functions efficiently, nor even the
    mathematical analysis of algorithms and the foundations of computing, but rather
    the techniques used to control the intellectual complexity of large software
    systems.
\cite{education:abelson_sussman__sicp}
\end{quotation}

\begin{quotation}
    Underlying our approach to this subject is our conviction that ``computer science'' is not a science and that
    its significance has little to do with computers. The computer revolution is a revolution in the way we
    think and in the way we express what we think. The essence of this change is the emergence of what might
    best be called procedural epistemology -- the study of the structure of knowledge from an imperative point
    of view, as opposed to the more declarative point of view taken by classical mathematical subjects.
    Mathematics provides a framework for dealing precisely with notions of ``what is.'' Computation provides
    a framework for dealing precisely with notions of ``how to.''
\cite{education:abelson_sussman__sicp}
\end{quotation}


\begin{quotation}
    Marvin Minsky and Seymour Papert formed many of our attitudes about programming and its place in our
    intellectual lives. To them we owe the understanding that computation provides a means of expression for
    exploring ideas that would otherwise be too complex to deal with precisely. They emphasize that a
    student's ability to write and modify programs provides a powerful medium in which exploring becomes a
    natural activity.
\cite{education:abelson_sussman__sicp}
\end{quotation}

Even the critics of the approach this text takes recognize its value and
depth: 

\begin{quotation}
    Abelson and Sussman have written an excellent textbook  which may start a
    revolution in the way programming is taught.
    Instead of emphasizing a particular programming language, they emphasize standard
    engineering techniques as they apply to programming.
    \cite{education:wadler__a_critique_of_abelson_and_sussman}
\end{quotation}

But not all is swell with this classic text. SICP does suffer from some
major deficiencies. Those are 2 according to \cite{education:felleisen__sicsc}:

\begin{quote}
    \begin{enumerate}
        \item ... sicp doesn’t state how to program and how to manage the design
        of a program. It leaves these things implicit and implies that students
        can discover a discipline of design and programming on their own

        \item SICP’s second major problem concerns its selection of examples and
        exercises. All of these use complex domain knowledge.
    \end{enumerate}
\end{quote}

Given those problems, I think it is best, specially for those without access to
an experienced tutor/colleague, to leave SICP for a later visit, armed with the
design tools that HTDP provides. Not only it will stress and test these tools,
but it will make the road more bearable and enjoyable.
