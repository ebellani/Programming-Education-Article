%1.Describe the problem
\section{Introduction}
The problem that I will cover in this article is the one of learning how to
think programmatically, that is, to program.

What I will be focusing is specially the demonstration of paths and technique's
a single aspirant programmer would use to grasp not only the minor surface
details of the craft, but its meaning and consequences, so that it could have a
chance at approaching complex and dynamic problems. The same holds true for a
teacher who wants to build a class for novice programmers.

There is a latent potential for an increase in the depth of the knowledge about
computation in the general population. You can observe for instance it by the
appreciation of computer games and graphics. Sure, not every gamer of graphic
designer has a desire for becoming a programmer, not even the majority. But one
can assume that from the enormous pool of people involved in those activities
there could be an significant category interested in computing itself.

There are a lot of benefits that leaning to program brings to an individual,
besides the possible monetary incentive. \cite{education:sussman__why_programming_is_a_good_medium}
Not only that, but this mastery could lead to an increase in critical thinking
and multiple levels of abstraction. As \cite{education:felleisen__htdp} states:

\begin{quotation}
    On one hand, program design teaches the same analytical skills as mathematics.
    But, unlike mathematics, working with programs is an active approach to
    learning. Interacting with software provides immediate feedback and thus leads
    to exploration, experimentation, and self-evaluation. Furthermore, designing
    programs produces useful and fun things, which vastly increases the sense of
    accomplishment when compared to drill exercises in mathematics. On the other
    hand, program design teaches the same analytical reading and writing skills as
    English. Even the smallest programming tasks are formulated as word problems.
    Without critical reading skills, a student cannot design programs that match the
    specification. Conversely, good program design methods force a student to
    articulate thoughts about programs in proper English.
\end{quotation}

But a lot on uncertainty exists as to the proper way to get educated. The
reasons are plenty, but an educated guess would probably involve the following
reasons:

\begin{enumerate}
    \item Too much disconnected information, sometimes conflicting with each other
    \item An emphasis on the final product and not the process in most of the
    ``official'' materials \cite{education:harvey__symbolic_programming_vs_AP_curriculum}
    \item A lack of explicit thinking tools for dealing with semantic
    complexity, and not syntactic detail\cite{education:felleisen__sicsc}
\end{enumerate}

%2.State your contributions
%...and that is all

The contributions I try to accomplish here are:

\begin{enumerate}
    \item A central and coherent roadmap one can use to become a competent
    programmer
    \item A review of the literature involved in the roadmap, pointing out their
    perceived strengths and weakness

    \item Some indications on where to go after the way is walked
\end{enumerate}

