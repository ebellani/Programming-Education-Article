%1.Describe the problem
\section{Introduction}
The problem that I will cover in this article is the one of learning how to
think programmatically, that is, to program.

What I will focus is specially the demonstration of paths and technique's
a single aspirant programmer would use to grasp not only the minor surface
details of the craft, but its meaning and consequences, so that it could have a
chance at approaching complex and dynamic problems. The same holds true for a
teacher who wants to build a class for novice programmers.

There is a latent potential for an increase in the depth of the knowledge about
computation in the general population. You can observe for instance it by the
appreciation of computer games and graphics. Sure, not every gamer or graphic
designer has a desire for becoming a programmer, not even the majority. But one
can assume that from the enormous pool of people involved in those activities
there could be an significant category interested in computing itself.

Besides this, a lot of benefits comes from learning to program, even if we
ignore the monetary incentive. \cite{education:sussman__why_programming_is_a_good_medium}
Not only that, but this mastery could lead to an increase in critical thinking
and multiple levels of abstraction, teaching analytical skills in an active and
concrete way. \cite{education:felleisen__htdp}


%2.State your contributions
%...and that is all

The contributions I try to accomplish here are:

\begin{enumerate}
    \item A central and coherent roadmap one can use to become a competent
    programmer by developing a taste for good programs

    \item A review of the literature involved in the roadmap, pointing out their
    perceived strengths and weakness

    \item Some indications on where to go after the way is walked
\end{enumerate}

